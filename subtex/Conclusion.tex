\section{Závěr}
%Všechno bylo skvělý a fungovalo to.
% co práce shrnuje atd co obsahuje
%%% co bylo cílem a jak se to povedlo
%%% potvrzení článku ze state of art o ML charakteristik 
%%% zlepšení datasetu (větší atd.) + nastavení parametru / testování dalších algoritmu / použití převodu malwaru na obrázek (viz stateofart)

Tato diplomová práce předkládá základní poznatky o analýze malwaru a to jak statické, tak dynamické analýze. Zabývá se také základními parametry struktury spustitelného souboru. Dále shrnuje základní mechanismy obrany malwaru, které se snaží maskovat záměry a funkce škodlivého kódu. To~je zásadní pro další vývoj analýzy malwaru, neboť je potřeba toto maskování detekovat potažmo i přes jeho přítomnost spolehlivě odhalovat malware. Dále se pak práce podrobněji zabývá analýzou statickou. Ke statické analýze pak nabízí přehled nejnovějších trendů v jejím využití, nezbytnou součástí je také výčet výhod a nedostatků tohoto konkrétního využití.

Na základě předloženého přehledu aktuálních znalostí na poli statické analýzy nabízí i vlastní řešení pomocí programu provádějícího statickou analýzu. Ten je dále testován, a to jak s vzorky legitimního softwaru, tak i se vzorky malwaru, přičemž důraz byl kladen na vzorky spustitelné v operačním systému s podporou formátu PE. Při analýze vlastním programem byly sledovány následující vlastnosti, jako jsou parametry spustitelného souboru, entropie, výskyt konkrétních skupin řetězců a nakonec výskyt konkrétních vzorů detekovaných pomocí nástroje Yara.

Výsledkem tohoto testování je soubor jednotlivých výstupů. Tyto výstupy jsou poté dále zkoumány.

Jako první byl vyhodnocován výstup parametrů spustitelného souboru. Legitimní software je běžně vyvíjen, jak pro 32 bitový, tak i 64 bitový operační systém a to zhruba v poměru 1:1, zatímco v případě malwaru převažuje výskyt 32 bitové varianty a to dokonce s 96~\%. Dalším parametrem byl podpis souboru digitálním certifikátem. Zde se ukazuje, že 33~\% běžných aplikací bylo podepsáno, naopak u malwaru to bylo 11~\%.  Byl zkoumán také typ PE souboru (dll knihovna nebo \emph{exe} spustitelný soubor). Posledním pro analýzu zajímavým parametrem byl počet sekcí spustitelného souboru, kde u malwaru bylo nalezeno průměrně menší množství sekcí než u legitimního softwaru.

Jedním z dalších výstupů pak byla entropie. Získané hodnoty entropie pro legitimní software měly normální rozložení a medián entropie běžných aplikací byl 6.41. Hodnoty entropie malwaru pak nebyly rozloženy normálně a jejich medián byl 7.08. Pomocí Mann-Whitneyova U-testu pak byla otestována shodnost hodnot entropie pro legitimní software a malware. Rozdíl mezi těmito dvěma skupinami byl statisticky významný.

Následně byly vyhodnocovány řetězce. Konkrétně se jednalo o email, IP adresu (jak IPv4, tak IPv6) a URL adresu. Zajímavým zjištěním bylo, že v případě malwaru byl výskyt emailové a url adresy nižší. V případě malwaru mohou tyto informace, které slouží často ke komunikaci s útočníkem, zůstat skryté díky různým obfuskačním metodám.

Podobně jako u řetězců nebyl výstup hodnotící výskyt knihoven průkazný. Na datech totiž můžeme jasně vidět, že se použité knihovny u běžného softwaru i malwaru velmi podobají. Proto~je potřeba zkombinovat tento výstup s výstupem obsahující seznam použitých metod, které jsou z těchto knihoven volány.

V případě výstupu funkcí byl pozorován znatelnější rozdíl mezi používanými metodami škodlivým kódem a běžnými aplikacemi. Malware častěji využíval metody, které slouží k dynamické práci s knihovnami a jsou často zneužívaný právě malwarem k injekci kódu.

Dále byl vyhodnocen výskyt použitých pravidel \emph{Yara}. Nejčastěji bylo detekováno pravidlo pro detekci 32 bitového PE souboru, použití grafického rozhraní a práce se systémovými soubory. U malwaru pak bylo v 53~\% detekováno packování a ve 44~\% inicializace pro obsluhu strukturovaných výjimek. Dalším zajímavým pravidlem, které bylo detekováno u malwaru v 22~\% byla snaha o získání vyššího oprávnění.

Tyto parametry byly následně také využity pro automatickou klasifikaci výstupu statické analýzy. Za tímto účelem byla využita technologie \emph{ML.NET}. 

Výsledky klasifikace malwaru nasvědčují, že úspěšnost použité detekce malwaru je velmi vysoká. U neškodlivého kódu pak byl v 15 \% testovaných  případů vyhodnocen testovaný legitimní software jako malware (false-positive). A proto by bylo vhodné datovou kolekci dále rozšířit o~vzorky nové a to jak malwaru tak běžného softwaru. Zároveň je však neustále potřeba hledat nové možnosti a parametry (jako naříklad využití zmíněné obrazové analýzy malwaru), protože~i~obrana škodlivého kódu vůči analýze se nezastavuje.

%Na základě těchto parametrů bylo následně využito technologie \emph{ML.NET}. Proto aby byl vytvořen automatický 
%Tato diplomová práce shrnuje aktuální možnosti v oboru statické analýzy malwaru. A zároveň předkládá vlastní řešení analýzy malwaru pomocí strojového učení s využitím technologie \emph{ML.NET}. 
%práce shrnuje akuální poznatky...., zároveň předkládá návrh vlastního postupu statické analýzy, který také testuje. Výsledky pak prokazují statistickou významnost mezi hodnotami entropií pro běžně používaný legitimní software a hodnotami entropií pro malware. Pokud by byl použit rozsáhlejší dataset.... + co by bylo možné zlepšit do budoucna
%možná ten článek o obrazové analýze?