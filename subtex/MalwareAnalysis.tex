% Analýza malware
\section{Analýza malwaru}

Analýza malwaru je proces při kterém se zkoumá chování vzorku. Předmětem takovéto analýzy je~zjistit, jak malware pracuje a jak jej případně eliminovat \cite{malwareanalysis_digitalguardian}. Výstup by měl obsahovat jednotlivé charakteristiky a funkcionality malwaru. Analýza podezřelých binárních souborů se~provádí v~bezpečném prostředí \cite{monnappaka2018}. Jak již bylo zmíněno, malware může nabývat mnoha podob jako například virus, červ, spyware nebo trojský kůň.

Analýza malwaru je prováděna v následujících případech \cite{malwareanalysis_digitalguardian}:

\begin{itemize}
    \item bezpečnostní incident v síti;
    \item za účelem zkoumání vzorku;
    \item indikace kompromitování systému.
\end{itemize}

Přístupy k analýze jsou dva a to pasivní a aktivní neboli statická a dynamická analýza. Popsány jsou v následujícím textu.

\subsection{Statická analýza}

Statická analýza je metoda, při níž není zkoumaný vzorek počítačového programu nebo jeho část spuštěn. Tento postup se využívá při analýze malwaru, testování aplikací, hledání zranitelnosti atd. Může se provádět buď ručně nebo pomocí specializovaného programu.

    Při statické analýze vzorku je nahlíženo na různé části programu jako jsou metadata, struktura souboru popřípadě zdrojový kód. 

%%Body k textu
%* bez spuštění
%* malware
%* testování aplikací / hledání zranitelností
%* možnosti:
%* * ručně
%* * automaticky
    %* hledání vzorů / signatur
    %* abstraktní interpretace
    %* vylepšení o neuronové sítě
    
% nastroje
% postupy
% metodika atd atd

%určení filetype (pe, elf..) (Y)
%vytvoření hashe (porovnání v db jestli už stejný soubor nemám)
%skenování antiviry - jestli už nějaká databáze neobsahuje
%extrakce stringu - ze souboru
%detekce packeru / obfuskace / komprese
%detaily spustitelného souboru (entry point atp) 
%klasifikace / širší analýza 

Průběh statické analýzy se může lišit případ od případu a nemusí být nutně provedeny všechny kroky \cite{monnappaka2018}. Obvykle však analýza malwaru začíná určením cílové platformy a to na základě informací získaných o spustitelném souboru a z něj. Například pokud podezřelý binární soubor v sobě obsahuje PE hlavičku, s velkou pravděpodobností se jedná o soubor určený pro operační systém Windows. 

%https://cs.qwe.wiki/wiki/List_of_file_signatures
Malware určený pro Windows často končí příponami .exe, .dll, sys. apod. Spoléhat se pouze na příponu však nelze. Útočník může použít nejrůznější způsoby jak příponu zakrýt tak, aby umožnil malwaru spuštění uživatelem. Proto by v kódu měla být použita signatura souboru, kterou známe také jako magická čísla\label{magic_numbers} \cite{magic_number}. Podpis souboru je totiž unikátní sekvence bytu v hlavičce souboru. Různé soubory mají různé signatury a lze je podle nich identifikovat \cite{file_signatures_2020}.

Dalším krokem je vytvoření unikátního otisku tzv. fingerprintu pomocí kontrolního součtu pro identifikaci daného vzorku a jeho porovnání s databází. Otisk souboru může být vytvořen pomocí libovolné hashovací funkce, která je používána v dané databázi. Aby bylo možné vzniklý otisk případně porovnávat s externími databázemi, je často vytvářeno více verzí pomocí více hashovací funkcí jako například MD5, SHA1 nebo SHA256. V případě, kdy by již vzorek existoval v databázi, není potřeba provádět další analýzu, protože její výsledek je již znám \cite{sikorski2012practical}.

Dále je pro zjištění zda má tento malware již známou signaturu případně fingerprint prováděn sken vícero anti-virovými programy \cite{sikorski2012practical}. Název signatury pak může sloužit k získání dalších informací o souboru a jeho vlastnostech. Pro tuto analýzu je možné použít například  nástroj VirusTotal, jenž obsahuje 7~desítek antivirů \cite{virustotal_howitworks}. 
    
Následuje extrakce obsažených řetězců (sekvence ASCII a UNICODE) obsažených v binárním souboru. Získání těchto obsažených řetězců může velmi usnadnit další analýzu podezřelého souboru. Například pokud se malware připojuje k CnC serveru, může obsahovat doménové jméno serveru, kde se CnC nachází. Dále mohou řetězce obsahovat například názvy souborů, které malware vytváří, URL adresy, IP adresy, informace o registrech atd. \cite{monnappaka2018}.

Protože je extrakce obsažených řetězců často klíčovým nástrojem jak získat informace o~malwaru a jeho funkcionalitě, snaží se proti ní útočnici svůj kód co nejvíce obrnit. A proto používají autoři škodlivého kódu packery (které provádí obfuskaci, šifrování a kompresi) tak, aby skryli malware před bezpečnostními experty. Tyto techniky totiž často velmi ztíží analýzu binárního souboru a oddálí tak detekci jejich malwaru.

Obsah hlavičky spustitelného souboru je dalším stěžejním bodem analýzy malwaru, obsahuje totiž důležité informace o zavedení do paměti a spuštění souboru, jeho struktuře, seznam potřebných knihoven atd. Tato analýza nám může například pomoci odhalit k čemu daný vzorek přistupuje (soubory, registry nebo třeba síť). Většina těchto operací je totiž prováděna pomocí API operačního systému a konkrétních knihoven. 

Následuje porovnání vzorku s jinými a jeho klasifikace. V případě malwaru jsou klasifikovány jednotlivé vzorky podle jejich charakteristik a podobností \cite{kaspersky_malware_classification}. Skupina, ve které se shlukují vzorky stejného typu, se nazývá rodina malwaru \cite{6107902}. 

Klasifikace do malwarových rodin je používána proto, že zatímco kontrolní součet souboru je skvělým nástrojem pro určení stejného vzorku, v případě malwaru se velmi často stává, že autor škodlivého kódu změní jen malou část kódu a tím změní celou hash \cite{monnappaka2018}.

\subsection{Dynamická analýza}
Narozdíl od statické analýzy je při dynamické analýze testovaný vzorek spuštěn. Vzorek se většinou testuje v kontrolovaném prostředí, v němž se vzorek spustí a sledují se změny, které provádí v systému. Vzhledem k tomu, že malware má uzavřený kód, je na něj při analýze nahlíženo jako na blackbox. Avšak většinou se při analýze malwaru kombinují obě techniky jak dynamická tak~statická analýza.

%%Body k textu
%* obdobně jako statická
%* program je spuštěn (musí být zkompilován / interpretován)
%* pak se sleduje jeho chování
%* vzhledem k tomu, že vetšinou má malware uzavřený kód, je na program nahlíženo jako blackbox
%* většinou se kombinují obě techniky
