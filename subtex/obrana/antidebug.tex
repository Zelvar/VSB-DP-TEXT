\subsection{Anti-debug}
\label{anti_debug}

Tento způsob ochrany brání ladění neboli debugovaní programu. Implementují ho často jak~legitimní aplikace, které potřebuji utajit své know-how, tak i tvůrci malwaru. Existuje mnoho způsobů jak přítomnost debuggeru detekovat a případně ho blokovat. Mezi ty nejpopulárnější patří následující \cite{sikorski2012practical}.

\paragraph*{Windows API}

Microsoft Windows obsahuje ve svém API několik zpusobů jak debugger detekovat. Některé funkce byly navrženy přímo pro detekci debuggeru, některé zase pro jiné účely, ale lze je pro detekci použít.

Nejjednodušším způsobem je použít metodu \emph{IsDebuggerPresent}. Tato metoda sleduje strukturu PEB, která obsahuje informace o aktuálním procesu, včetně pole \emph{IsDebugged}. Pokud je~toto pole nastaveno na nulu, debugger není přítomen. V opačném případě je program spuštěn v~debug režimu.

Obdobným způsobem můžeme použít metodu \emph{CheckRemoteDebuggerPresent}, která funguje na podobném principu. Rozdíl spočívá v tom, že tato funkce byla navržena ke sledování debuggu cizího (vzdáleného) procesu. Lze ji však nastavit tak, aby plnila i tento úkol.

Další možností je funkce \emph{NtQueryInformationProcess}. Metoda opět slouží k získání informací o požadovaném procesu. Prvním parametrem funkce je ukazatel na proces. Dalším parametrem je pak možné nastavit typ získané informace. 

Alternativním řešením bez sledování struktury PEB může být \emph{OutputDebugString}. Funkce slouží k odeslání řetězce na výstup debuggeru. Pokud však debugger nebude přítomen a chybový kód bude nastaven pomocí funkce \emph{SetLastError} bude po zavolání \emph{OutputDebugString} na výstupu chybový kód funkce \emph{OutputDebugString}. V případě, že by debugger přítomen byl, na výstupu bude chybový kód, který byl nastaven přes funkci \emph{SetLastError}.

\paragraph*{Detekce chování debuggeru}

Další variantou této ochrany je detekce základní funkce debuggeru a to breakpointu \cite{deepinstrinct_antidebug}. Breakpoint slouží analytikovi k zastavení kódu a funguje na principu vložení instrukce \emph{INT 3} do kódu. Instrukce \emph{INT 3} slouží k vyvolání breakpointu neboli přerušení. Malware však může skenovat sám sebe a hledat tuto instrukci v kódu. Případně vytvářet za~běhu kontrolní součet nebo hash aby zjistil, zda nebylo zasaženo do kódu.


\paragraph*{TLS callback}
%https://www.deepinstinct.com/2017/12/27/common-anti-debugging-techniques-in-the-malware-landscape/
%https://docs.microsoft.com/cs-cz/cpp/parallel/thread-local-storage-tls?view=vs-2019
TLS je místní paměťový prostor v rámci jednoho vlákna, kde může dané vlákno ukládat svá data \cite{msdocs_tls}. Tato vlastnost může být zneužita tak, že ještě před vstupním bodem programu, kdy je inicializováno vlákno aplikace, se zavolá potřebná funkce \cite{deepinstrinct_antidebug}.


\paragraph*{Zneužití zranitelnosti debuggeru}
%https://exchange.xforce.ibmcloud.com/vulnerabilities/16711
Stejně jako každý software i debuggery mohou obsahovat zranitelnosti. Autoři malwaru si to samozřejmě uvědomují a této skutečnosti zneužívají. Příkladem může být chyba známého a používaného debuggeru OllyDbg ve verzi 1.1 \cite{CVE-2004-0733}, která umožňovala aplikaci shodit pomocí formátovacího řetězce, zaslaného přes Windows API metodou \emph{OutputDebugString}.