\subsection{Komprese}

Cílem komprese spustitelných souborů je zmenšit celkovou velikost kódu a sekcí s daty. Původní spustitelný soubor se nahradí souborem novým. Tento nový soubor obsahuje program pro dekompresi a původní spustitelný soubor jež je komprimovaný.  Pří spuštění se původní soubor nejdřív dekomprimuje do paměti a následně se spustí \cite{golchikov_2002}.

Běžné formáty spustitelných souborů, jako PE, ELF atp. v základu kompresi nepodporují. Proto v době, kdy byl nedostatek paměti a malá velikost spustitelných souborů byla tedy nutností, začala tak vznikat různá řešení pro kompresi spustitelných souborů.

V současnosti se však komprese využívá spíše z důvodu možnosti skrýt obsah spustitelného souboru bez jeho dekomprese.

%https://reverseengineering.stackexchange.com/questions/14288/what-is-executable-compression
%https://patentimages.storage.googleapis.com/90/80/a7/7bd0343cf32f90/US20020112158A1.pdf


% DEKOMPRESE?