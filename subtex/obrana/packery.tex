\subsection{Packery}
\label{packers}

Při použití packingu dochází k zabalení původního programu nebo jeho části a to pomocí packeru nad binárními daty. Může docházet k zašifrování nebo kompresi. Packer nahradí původní program nebo jeho část tzv. unpackerem. Při spuštění kódu se nejprve provede tzv. unpacking do paměti a následně se kód spustí \cite{diff_packers}. Tvůrci malwaru využívají tuto techniku packování velmi často, za~účelem ztížení a časové prodloužení analýzy kódu reverzním inženýrem \cite{HoonKang2011}.

% Kompresory, bundlery, kodéry, protektory

%% EXAMPLE IMAGE

Při svém procesu mohou zároveň provádět kompresi, obfuskaci, šifrování nebo přidat jinou funkcionalitu pro ztížení reverzní analýzy \cite{packers_2010}. Druhy packerů si popíšeme v následujícím textu, součástí bude porovnání několika packerů viz tabulka č. \ref{table:packeryTabulka}. Některé packery, jež jsou porovnávaný jsou veřejně dostupné a mají také otevřený zdrojový kód (UPX, Obfuscator, ConfuserEX). Otevřený kód může nahrávat reverzním analytikům, protože snáze zjistí jak packer funguje a mohou tak tedy kód případně snáze unpackovat. Existují však také komerční packery, které otevřeným kódem nedisponují a je těžší je analyzovat.

%https://gironsec.com/code/packers.pdf
%https://www.boxedapp.com/exe_bundle.html
%https://www.security-portal.cz/clanky/seznamte-se-%E2%80%93-morfismy-oligomorfismus-polymorfismus-metamorfismus

%%
\paragraph*{Rozšiřující}

\subparagraph*{Kompresory} Tento druh packeru slouží primárně k snížení velikosti spustitelného souboru. Před spuštěním se provede dekomprese do paměti a soubor se spustí \cite{diff_packers}.

\subparagraph*{Protektor} Cílem protektoru je ztížit analýzu kódu různými metodami jako anti-debug, anti-vm apod. a ochránit tak kód vůči reverzní analýze \cite{packers-malwarbytes}.

\subparagraph*{Kryptor}  Dalším druhem packeru je kryptor, jež provádí šifrování originálního souboru. Před spuštěním je nejprve dešifrován a následně spuštěn \cite{packers-malwarbytes}.

\subparagraph*{Bundler} Tato metoda packování umožňuje zabalit do jednoho souboru všechny potřebné soubory. Program se tedy na první pohled tváří, že neobsahuje žádné externí knihovny apod. \cite{diff_packers}. 

%%
\paragraph*{Transformující} %https://www.security-portal.cz/clanky/seznamte-se-%E2%80%93-morfismy-oligomorfismus-polymorfismus-metamorfismus

\subparagraph*{Virtualizátor} Virtualizátory převádí původní spustitelný kód do jazyku virtuálního stroje, který následně provádí vestavěný virtuální stroj \cite{diff_packers}. 

\subparagraph*{Mutátor} Převádí instrukce na alternativní (v rámci stejné platformy). Využívá se oligomorfismu \cite{diff_packers}.


%https://blog.malwarebytes.com/cybercrime/malware/2017/03/explained-packer-crypter-and-protector/
%https://www.mcafee.com/blogs/enterprise/malware-packers-use-tricks-avoid-analysis-detection/
%https://reverseengineering.stackexchange.com/questions/1779/what-are-the-different-types-of-packers

%https://archive.codeplex.com/?p=netdeob0
%https://en.wikipedia.org/wiki/Executable_compression

\noindent
\begin{table}[!ht]
    \centering
    \caption{Výběr některých packerů}
    \label{table:packeryTabulka}
    
    \begin{tabular}{|l|l|c|c|c|c|c|}
    \hline
        Název & Licence & x86-64 & Komprese & Obfuskace & Šifrování & Jiná \\
		\hline
		\hline
        UPX & GPL & \checkmark & \checkmark &  & & \\  \hline
        
        ASPack & Proprietární & x86 & \checkmark &  & & \\  \hline
        ASProtect & Proprietární & \checkmark & \checkmark &  & \checkmark & \checkmark \\  \hline
        Enigma Virtual Box & Proprietární & \checkmark & \checkmark & \checkmark & & \checkmark \\  \hline
        
        \hline
        Obfuscar & MIT & .NET & \checkmark & \checkmark & & \\ \hline
        ConfuserEx & MIT & .NET & \checkmark & \checkmark & \checkmark & \checkmark \\ \hline
    \end{tabular}
\end{table}

\subsection{Unpacking}
\label{unpackers}

Unpacking je proces, při kterém dochází k obnově původních zdrojových dat (kódu) z programu, jež byl zabalen jedním nebo vícero packery \cite{HoonKang2011}. Existují tří různé způsoby jak získat původní kód \cite{4639028}. A to buď pomocí ručního, statického nebo dynamický unpacking \cite{generic_unpacker}.

\paragraph*{Ruční unpacking}

Při ručním unpackingu reverzní analytik zkoumá jednotlivé vrstvy algoritmu, jímž byl kód šifrován, komprimován apod. A ručně se snaží o obnovu původních dat. Tato~technika je však velmi časově náročná, vyžaduje hlubší znalosti nižších vrstev OS a také znalost assembleru \cite{4639028}. 

\paragraph*{Statický unpacking}

Metoda statického unpackingu je vlastně způsob jak zautomatizovat unpacking známých packerů jako například kompresor UPX. Jde o rutinní operace, které provádí dekompresi, dešifrování apod. A umožňují tak snadno získat původní kód. Autoři malwaru však mohou mírně upravit packer nebo použít svůj vlastní a původní unpacker již nebude funkční \cite{4639028}. 

Tuto metodu také používají antivirové společnosti ve svém softwaru, aby urychlili detekci nových neznámých vzorků malwaru \cite{4639028}.

\paragraph*{Dynamický unpacking}

Zatímco statický unpacking se snaží nahradit proces packeru, jež~se stará o rozbalení a spuštění původní aplikace. Dynamický unpacker nechá rozbalení na původním programu. Unpacker nejdříve spustí zabalený program, nechá jej až se rozbalí do~paměti. A pak se snaží získat rozbalený kód z paměti a uložit do souboru \cite{generic_unpacker}.